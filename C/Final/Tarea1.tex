\documentclass[11pt,letterpaper]{article}
\usepackage{amsmath}
\usepackage{amsfonts}
\usepackage{amssymb}
\title{Tarea 1B}
\author{Sebasti\'an Alejandro Gallardo D\'iaz (201410006-k)\\Javier Ignacio Ortiz Gonzalez (201473061-6)\\Eduardo Francisco Pozo Vald\'es (201473040-3)}
\begin{document}
\begin{titlepage}
\maketitle
\end{titlepage}
\section{Tiempo:}Se termino el c\'odigo en 8 horas aproximadamente.
\section{Inactividad de los integreantes:}Todos colaboramos en la realizaci\'on de la tarea. Sebasti\'an fue el que dio las ideas principales del c\'odigo mientras todos tratamos con el problema de pasar el pseudoc\'odigo con sus ideas a C++.
\section{Principales dificultades:}
$\circ$ Pensar de manera recursiva.\\
$\circ$ Buscar formas alternativas de retornar arreglos que se necesitaban dentro del c\'odigo.
\section{Problemas del c\'odigo:}
Alta complejidad temporal debido a que se eval\'uan todas las posibilidades por backtracking recursivo. El arbol de posibilidades no est\'a podado, es decir, no se eliminan las posibilidades que son iguales por teor\'ia de conjuntos. 
\end{document}
